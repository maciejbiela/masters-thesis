\paragraph{}
The goal of this thesis was to use image processing techniques for identification of foundry details. An algorithm has been created for comparing images of splints and detecting whether two of them represent the same object or two different objects. Despite having the images taken by a smartphone with varying lighting conditions, a satisfactory result has been achieved. Some hints for possible solutions were also provided for the preprocessing phase. The thesis presents ideas for improvements that should be considered before production implementation.

\paragraph{}
Plentitude of runs, adjustments and configurations of the algorithm over the input data has led to a hypothesis that significantly better accuracy can be achieved when the object are placed on the images under the same angle. This hypothesis has been later confirmed with results and statistical tests in chapter \ref{chap:results}.

\paragraph{}
Further on, the thesis has also provided an overview of many image processing aspects and transformations heavily used within the developed algorithm. Both the theory and also some examples were presented. All described algorithms are also available in the OpenCV library that served as the fundamental library for implementing the algorithm of splints comparison.

\paragraph{}
Last but not least, the usage of OpenCV - described before; NumPy and Matplotlib - the Python libraries also extensively used, was provided with some simple code samples that can be used as a quick starters.