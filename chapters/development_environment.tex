\lstdefinestyle{mystyle}{
    breakatwhitespace=false,         
    breaklines=true,                 
    captionpos=b,                    
    keepspaces=true,                 
    numbers=left,                    
    numbersep=15pt,                  
    showspaces=false,                
    showstringspaces=false,
    showtabs=false,                  
    tabsize=2
}
 
\lstset{style=mystyle}

\section{OpenCV}
\paragraph{}
OpenCV (\url{https://opencv.org/}) is an open source computer vision library available under BSD licence. Its' alpha version was released in 1999 by an Intel Corporations' employee with hopes of a quicker and broader evolution of computer vision and artificial intelligence development. The library is primarily written in C and C++ and runs under any modern operating system. Apart from the core library, interfaces in other languages (such as Python, Java, etc.) are actively developed. Other major contributors to OpenCV include companies such as Google, Itseez and Arraiy.\cite{learning-opencv-3}
\paragraph{}
The library has been widely adopted both in business and in scientific research efforts. This popularity is a consequence of great computational efficiency that is achieved by optimized C++ code and the ability to make the most out of multicore processors. The liberal license also is a reason for it being so well-received. A commercial product can be freely built using OpenCV without any obligation to open-source it.
\paragraph{}
As computer vision goes hand-in-hand with machine learning, a fully-fledged ML module is also a part of OpenCV that is focused on clustering and statistical pattern recognition. This module mostly aims to be useful for computer vision tasks, but could also be used for solving other machine learning problems.
\paragraph{}
OpenCV since its' inception was developed to make computer vision infrastructure available to masses, to advance the research by providing core building blocks in order not to reinvent the wheel every single time. It also aimed to propagate the knowledge, standardize the development and advance vision-based applications. 

\section{OpenCV Python bindings}
\paragraph{}
Python is a general purpose programming language with rich standard library. It's main focus has always been both readability and clarity of the source code. It cooperates well with different programming paradigms, including object-oriented, imperative and, to a lesser degree, functional style. Python is a dynamic language that is often used for scripting purposes. It is slower in comparison to languages like C or C++ but can itself be easily extended with both of those. Simple Python wrappers can be applied to code written in C/C++ and this process allows to create highly efficient Python modules.

\section{NumPy}
\paragraph{}
OpenCV Python bindings extensively use NumPy (\url{http://www.numpy.org/}). NumPy is a fundamental package for scientific computing and numerical operations with a MATLAB-like syntax. It's main object is the multidimensional array. It is a grid of values, all of the same type (often numbers), and indexed by a tuple of non-negative integers. Dimensions are called \textit{axes} and the number of axes defines the \textit{rank}.
\paragraph{}
\texttt{ndarray} (known also by the alias of \texttt{array}) is the array class in NumPy. It is important to remember that \texttt{numpy.array} is different from the class from Python's standard library (\texttt{array.array}). The latter one is designed for handling only one-dimensional arrays and its' functionality is also limited when compared to NumPy.

\paragraph{}
Here I will write about Python, anaconda, Jupyter, Docker...

\begin{lstlisting}[language=Python, caption=Python test code]
import cv2
import numpy as np

img = cv2.imread('some_cool_image.jpg')
\end{lstlisting}

\todo{matplotlib}
\todo{pipenv}